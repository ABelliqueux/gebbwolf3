\documentclass[book.tex]{subfiles}
\begin{document}
For the 20 years anniversary, John Carmack revisited and played the game with commentaries. Here is the transcript :


   
\section{The engine (4:00 mark)}

The two previous 3D games Hovertank and catacombs 3D were done in an object space rendering where it drew limited polygons. They were one-dimensional in terms that they were just line segments that were restricted on axises.\\
\par
We had something that resembled a polygon rasterizer and a polygon clipper and those were both done in four to six weeks a piece. I had really quite a bit difficulty with it. Going back in time twenty years, there weren't all of the references in existence, books and tutorials on this subject.\\
\par
I was having a hard time getting some of that stuff to be as robust
and reliable as it needed to be. You could get a few freak out cases
in Catacombs 3D and one of my real goals was to simplify it enough
that it would be really rock solid and robust.\\
\par

The fact that Wolfenstein 3D came out with a speedup have more to do with the poor quality of my previous implementation because when Wolfenstein  needed to gain some speed on much poorer platforms like the Super Nintendo I went back to more of a rasterization approach with BSP trees rather than ray casting.\\
\par

Wolfenstein did wind up being both more efficient and more robust than my previous implementations but it was all wild-west for me back then :I was figuring it  all out as I went along and there were a lot of things to kinda look at.\\
\par

As example I remember that we had the first level running at about three months in. We followed it up with spear destiny on there before moving to doom.\\
\par
It was very short amount of time. We leveraged the toolset that we were using to create the 2d games the Commander Keen series where we had a good tile editor that John Romero had developed and since the Wolfenstein maps were basically simple tile maps we were able to make things happen really quickly on that and also the maps were just so quick to create: we had several maps in shipping products that were literally done in one day somebody would go scrub out a map we played a bunch of times tweak things a little bit and it would go in the project there. and it's  always interesting looking back  at  the games there where you look at the source code and it all fits in one directory its one handful C files and a few ASM files and there's just not that much to it when we look at what we're doing today and it seems that you know we can't throw a dialog up on the screen without invoking ten different frameworks and fifty thousand lines of code.\\



\section{Modding}

The thing that stands out a lot was the first (and it influenced a lot
of what iD Software did afterwards) was that people kinda tweaking with
the projects. People figuring out how to unpacked the levels. That 
wasn't straightforward because We were trying to fit on floppy disks 
at this time and I had developed all this compression technology course:
The funny thing is in hindsight I independently reinvented LZSS coding 
(in very ad hoc ways). \\
\par
People figured all these out extracted everything and started making 
character editors and  level editors. Neither of those were designed 
to be done and they weren't straightforward. The characters were in 
this column packed format that made it more efficient to draw without 
transparency test but made it really difficult to figure out what 
these original pictures looked like. \\
\par
These people started doing these really neat things with it. Once we 
just recognized that there was a large body of people that wanted to 
do this, it influenced a lot of our future decisions in Doom and Quake 
about making it actually easy and straightforward and encouraging 
people to undertake modding.\\
\par
I have cleared recollections to this day and especially at that time 
about thinking back to when I was getting into gaming when I was a 
teenager about how I wish that I could have that kinda access to the 
inside and guts of the games that I played. I can remember breaking 
out the Apple II sector editors to give myself lots of gold in 
Ultima III and that type of stuff and wishing that I had the ability 
to look at the source code for those old titles and being able to 
make that type of things come true as we were later able to later 
release  the full source code as well as the modding tools for the 
games has been something that am really proud of in my career.\\




\section{Texture mapping (13:30 mark)}


Normally when you're decent distance from things everything smoothly 
expands on there but if you notice when you get up close and start 
going off the edge of the screen you start getting more quantization 
in that. That is a result for the fact that the core graphics techno-
logy behind the texture mapping in this timeframe was compiled 
scalars: There is actually a different little section of assembly 
language code that was programmatically generated to draw a 64 tall 
piece of graphics:\\

\par
\begin{lstlisting}[breaklines=true,breakindent=0em]
two  pixels tall 
four pixels tall 
six  pixels tall

and so on...
\end{lstlisting}

all the way up to the full height of the screen but it started taking 
up too much memory and would run out in a 640k system if I let them 
stretch all the way up to the largest possible scale you get there in 
steps by one so it started taking some shortcuts and saving a little 
bit has it got bigger\\
   

\end{document}
