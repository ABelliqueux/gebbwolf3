\documentclass[book.tex]{subfiles}
\begin{document}
 \begin{fancyquotes}
    When I ported Wolf to the SNES, the ray casting performance cost was too much, so I had to make a new wall span renderer.  Learning about BSP trees allowed me  to much more accurately resolve the culling challenges, and it worked out ok, leading the way to the Doom renderer.\\
\\
Many years later, I made a very similar programming tradeoff for the mobile BREW version of Doom RPG.  The J2ME (java) Doom RPG looked like Wolfenstein, with a tile map world, textured walls, and solid color floor and ceiling, but it was done with a quite nice wall span renderer.  I had learned a thing or two since writing Wolf.  For the ARM native code BREW version, I wanted to add texture mapped floors and ceilings with per-tile texture choices.  I turned the tile maps into polygon windings and started writing a full texture mapped clipped polygon rasterizer for them, but I only had a couple days to work on the mobile renderer, and it became clear that I wasn’t going to be able to deliver a really solid implementation in that time.\\
 \\
The solution was to sacrifice performance for implementation simplicity.  Instead of trying to fill in just the empty pixels around the walls with floor / ceiling textures, I completely textured the entire screen with floor and ceiling textures before drawing the walls on top of them.  With floor / ceiling symmetry and fixed texture sizes, it was pretty fast even with per-pixel tile map lookup, but most importantly it was rock solid, crack free, and completed in the window of time I had allotted to it.\\
\\
\textbf{John Carmack - Programmer}
\end{fancyquotes}
\end{document}