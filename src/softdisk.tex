Wolfenstein 3D was not the first FPS the team produced. Back when they still had obligations to Softdisk, they worked on two games: Hovertank 3D and Catacomb 3D.\\
\par
\section{Hovertank 3D}
Hovertank 3D (a.k.a Hovertank, Hovertank 3-D or Hovertank One) is a vehicular combat game published by Softdisk in April, 1991. Set during a nuclear war, the game puts the player in control of a tank. The goal of the game is to kill mutated monsters and rescue survivors. John Carmack's research for the game's engine took six weeks, two weeks longer than any engine he wrote before. There was no texture mapping on the walls or the floor/ceiling and the 3D engine was targeted at EGA (16 colors). The pace of the game was slow and the game had no music. The digitized audio effects were just Romero making noises into a microphone! \\
\par

\section{Catacomb 3D}
Catacomb 3D (a.k.a Catacomb 3-D: A New Dimension, Catacomb 3-D: The Descent, and Catacombs 3) was released in November 1991. The game put the player in the shoes of a magician fighting goblins and orcs. The engine was improved with texture mapping for the walls. While still using EGA and in 16 colors, the game looked much better than Hovertank 3D thanks to improved assets. The pace of the game was also set to be faster. Note that players could destroy walls with fireballs.\\
\par


\begin{minipage}{\textwidth}
\ref{hovertank3d_screenshot}
\begin{figure}[H]
\centering
\scaledimage{0.9}{Hovertank_3D_title_screen.png}
\end{figure}

\begin{figure}[H]
\centering
\scaledimage{0.9}{Hovertank_3D_screen.png}
\end{figure}
\end{minipage}


\begin{minipage}{\textwidth}
\ref{catacomb3d_screenshot}
\begin{figure}[H]
\centering
\fullimage{Catacomb_3-D_The_Descent_title_screen.png}
\end{figure}

\begin{figure}[H]
\centering
\fullimage{Catacomb_3-D_The_Descent_screenshot.png}
\end{figure}
\end{minipage}

