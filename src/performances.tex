\section{Performance}
With all the tricks and optimizations described, how well did the game run on most machines? With most of the computers of this era gone, it is hard to tell. But thanks to the personnal collections of Jim Leonard and Foone Turing, and a special version of Wolfenstein 3D displaying frames per seconds, we were able to come up with the numbers in figure \ref{perf_summary}.\\
\par
By far the most important component after the CPU was the VGA card. Cirrus Logic RAM was optimized for fast write (and slow read) which explains the huge performance boost compared to other cards. A 386DX-40 with a bad VGA card can be brought to the same level as a 386SX-16 with a good card. Likewise, upgrading the VGA card could double performances on a 386DX-40.\\ 
\par

\par
\begin{figure}[H]
\centering
\begin{tabularx}{\textwidth}{ c  c c  c  c c c }
  \toprule
  \textbf{CPU} & \textbf{Mhz} & \textbf{Cache} & \textbf{Audio} & \textbf{VGA Card} &\textbf{Bus} & \textbf{Avg FPS} \\ \bottomrule

286 & 6mhz  & 0k  & AdLib & Hercules VGA & 8 &   5  \\ \bottomrule
286 & 12mhz  & 0k & AdLib & Paradise PVGA1A & 16 &  7  \\ \bottomrule
286 & 25mhz  & 0k & ESS1868 & CirrusLogic 5420 & 16 &  19  \\ \bottomrule
386SX &	16mhz	&512k		& No & Oak Technology OTIVGA 	& 16	& 10 \\
%386SX &	16mhz	&512k		& No & ATI TVGA8816CSC &	16	& 10 \\
386SX &	16mhz	&512k		& No & ATI VGA Wonder	& 16	& 10 \\
386SX &	16mhz	&512k		& No & ATI VGA Wonder &	8	& 10 \\
386SX &	16mhz	&512k		& No & Tseng Labs et300ax	& 8	& 11 \\
386SX &	16mhz	&512k		& No &  Trident Tvga8900c	& 16 &	12 \\
386SX &	16mhz	&512k		& No & Headland tech GC208-PC	& 16 &	13 \\
386SX &	16mhz	&512k		& No & Cirrus Logic AVGA3M-C03	& 16	& 14 \\ \bottomrule

386SX &	40mhz	&0k		    & No & ATI VGA Wonder	& 16	& 16 \\
386SX &	40mhz	&0k		    & No & ATI VGA Wonder	& 8	& 16 \\
%386SX &	40mhz	&0k		    & No & ATI TVGA8816CSC	& 16	& 16 \\
386SX &	40mhz	&0k		   & No & Tseng Labs et300ax	& 8	& 17 \\
386SX &	40mhz	&0k		   & No & Oak Technology OTIVGA 	& 16	& 17 \\
386SX &	40mhz	&0k		& No &  Trident Tvga8900c &	16	& 21 \\ 
386SX &	40mhz	&0k		& No & Headland tech GC208-PC	& 16	&	24 \\
386SX &	40mhz	&0k		& No & Cirrus Logic AVGA3M-C03	& 16	&	26 \\ \bottomrule

%386SX &	40mhz	&0k		    & SB &	ATI TVGA8816CSC	& 16	& 15 \\
386SX &	40mhz	&0k		    & SB	& ATI VGA Wonder	& 8	& 16 \\
386SX &	40mhz	&0k		   & SB	& ATI VGA Wonder	& 16	& 17 \\
386SX &	40mhz	&0k		   & SB	& Oak Technology OTIVGA 	& 16	& 17 \\
386SX &	40mhz	&0k		   & SB	& Tseng Labs et300ax	& 8	& 18 \\
386SX &	40mhz	&0k		& SB	&  Trident Tvga8900c	& 16	&	21 \\
386SX &	40mhz	&0k		& SB	& Headland tech GC208-PC	& 16	&	21 \\
386SX &	40mhz	&0k		& SB	& Cirrus Logic AVGA3M-C03	& 16	&	25 \\ \bottomrule

%386DX &	40mhz	&128kb		& No & ATI TVGA8816CSC	& 16	& 20 \\
386DX &	40mhz	&128kb		& No & Tseng Labs et300ax	& 8	& 21 \\
386DX &	40mhz	&128kb		& No & ATI VGA Wonder	& 16	& 21 \\
386DX &	40mhz	&128kb		& No & ATI VGA Wonder	& 8	& 21 \\
386DX &	40mhz	&128kb		& No & Oak Technology OTIVGA  & 	16	&	22 \\
386DX &	40mhz	&128kb		& No &  Trident Tvga8900c	& 16	&	26 \\
386DX &	40mhz	&128kb		& No & Headland tech GC208-PC	& 16	&	32 \\
386DX &	40mhz	&128kb		& No & Cirrus Logic AVGA3M-C03	& 16	&	33 \\ \bottomrule

%386DX &	40mhz	&128kb	   & SB	& ATI TVGA8816CSC	& 16	& 17 \\ 
386DX &	40mhz	&128kb		& SB	& ATI VGA Wonder	& 16	& 20 \\
386DX &	40mhz	&128kb		& SB	& ATI VGA Wonder	& 8	& 20 \\
386DX &	40mhz	&128kb		& SB	& Tseng Labs et300ax	& 8	& 21 \\
386DX &	40mhz	&128kb		& SB	& Oak Technology OTIVGA  & 	16	&	21 \\
386DX &	40mhz	&128kb		& SB	&  Trident Tvga8900c	& 16	&	26 \\
386DX &	40mhz	&128kb		& SB	& Headland tech GC208-PC	& 16	&	32 \\
386DX &	40mhz	&128kb		& SB &	Cirrus Logic AVGA3M-C03	& 16	&	34 \\ \bottomrule
\end{tabularx}
\caption{Average fps per machine.}
\label{perf_summary}
\end{figure}



To compensate for power differences, the engine can reduce the 3D canvas to lower the number of rays cast and the number of pixels to render. The maximum is 304 rays resulting in 46208 pixels to render (304*152) and the minimum is 64 rays resulting in 2432 pixels to render (64*38).\\
  \begin{figure}[H]
\centering
 \fullimage{adjust_view/fps16.png}
 \end{figure}
 \par
 In the previous screenshot the game is running at maximum resolution: 304 rays, resolution of 304x152. This 386SX-16Mhz achieved 16 frames per second.\\
 \par
 In the next two screenshots the framerate improves as 3D canvas size is reduced. 224 rays and a resolution of 224x112 raises the rate to 20 fps. 64 rays and a resolution of 64x38 brings it up to 52 fps.\\
 

   \begin{figure}[H]
\centering
 \scaledimage{0.87}{adjust_view/fps20.png}
 %\caption{Max view: 224 rays, resolution of 224x112}
 \end{figure}
 \par

   \begin{figure}[H]
\centering
 \scaledimage{0.87}{adjust_view/fps52.png}
 %\caption{Min view: 64 rays, resolution of 64x38}
 \end{figure}
 \par
\bu{Trivia :} In 1994, 3D Realms published Rise Of the Triad (a.k.a ROTT), directed by Tom Hall. The game uses a highly-modified\footnote{Textured floors and ceiling, different wall heights, stairs, trampolines, and even jumps.} Wolfenstein 3D engine and also features the same window size adjustment system. True to the humor found in early 90s video games, it features an Easter egg teasing the player to buy a 486 when the lowest setting is chosen.
    \begin{figure}[H]
\centering
 \fullimage{adjust_view/rott.png}
 \end{figure}
 \par
