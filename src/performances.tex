\section{Performances}
With all the tricks and optimization described, how well did the game run on most people desk? DOSBox allows to emulate old CPU and gives the following figures\footnote{The original Wolfenstein3D did not show the frame rate. An altered version of the engine from http://thandor.net/article/10 featuring framerate running on DOSBOX was used to get these numbers. DOSBOX settings were chosen with 1000 cycles = 1 MIPS (confirmed with MIPS.COM DOS executable.)}.

\begin{figure}[H]
\centering
\begin{tabularx}{\textwidth}{ X X X }
  \toprule
  \textbf{CPU} & \textbf{Frequency (Mhz)} & \textbf{Average framerate} \\ \bottomrule
 286 & 6 & 6 \\
 286 & 8 & 10 \\
 386SX & 16 &  16 \\
 386SX & 33 & 30 \\
 386DX & 33 & 50 \\
 386DX & 40 & 60 \\ \bottomrule
\end{tabularx}
\end{figure}

To compensate for the differences of power, the engine allows to reduce the 3D canvas which reduces the number of rays to cast and the number of pixels to render on screen.
\par
Max: 304 rays resulting in 46208 pixels to render (304*152).\\
Min:  64 rays resulting in 2432 pixels to render (64*38).\\

  \begin{figure}[H]
\centering
 \fullimage{adjust_view/fps16.png}
 \end{figure}
 \par
 Above the game running in max view: 304 rays, resolution of 304x152. The simulated 386SX-16Mhz is achieving 16 frames per seconds.

   \begin{figure}[H]
\centering
 \fullimage{adjust_view/fps20.png}
 \caption{Max view: 224 rays, resolution of 224x112}
 \end{figure}
 \par

   \begin{figure}[H]
\centering
 \fullimage{adjust_view/fps52.png}
 \caption{Min view: 64 rays, resolution of 64x38}
 \end{figure}
 \par

In 1994, Rise Of the Trial (a.k.a ROTT), also directed by Tom Hall was publisehd by 3D Reams. It used Wolfenstein 3D engine and also features the window adjustment system. In the line of the humor of the early 90s video games, it features an easter egg when the player chose the lowest setting, teasing them to buy a 486.
    \begin{figure}[H]
\centering
 \fullimage{adjust_view/rott.png}
 \end{figure}
 \par