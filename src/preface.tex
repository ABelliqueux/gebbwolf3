\documentclass[book.tex]{subfiles}
\begin{document}

For the past 10 years, I have been writing articles explaining the internals of game engines. It all started back in 1999 when I downloaded the freshly open sourced code of \mbox{Quake} and eagerly opened it with \codeword{Visual Studio .NET}. After a few days I deleted \codeword{quakesrc} folder, unable to make sense of anything and discouraged.

\bigskip

A few years later I came across the legendary Graphics Programming Black Book by Michael Abrash. His articles gave away the big picture of Quake engine and detailed BSP\footnote{Binary Space Partition, Chapter 59 in Michael Abrash Graphic Programming Black Book.} and PVS\footnote{Probably Visiblet Set, Chapter 64 in Michael Abrash Graphic Programming Black Book.} compression techniques. Knowing what to expect, I went back to the code and understood it to the deep down.

\bigskip

I thought many other programmers may be like me: Capable but discouraged by apparent complexity. So I started to write ``source  code reviews'' and published them on 
\href{http://fabiensanglard.net}{\textit{fabiensanglard.net}}. Over the years, I have published more than fifty articles, selecting legendary games such as Doom, Quake or Out Of This World. I would open the engine, explore the subsystems, understand the overall architecture and draw a map that hopefully sparkled interest and encouraged other adventurous programmers.

\bigskip

Sharing my knowledge was a rewarding experience: Not only the feedback from readers was overwhelmingly positive, to try to explain something in simple terms was also a way to make sure I mastered a topic. It also allowed me to become a better teacher and learn to rely extensively on good drawings. A picture
is worth more than a thousand words\footnote{"Code: The Hidden Language of Computer Hardware and Software" by Charles Petzold is a superb example.}. Those skills proved invaluable in my career which eventually lead me to Google.

\bigskip

Eventually I decided to take my articles to the next level and came up with this book which I hope will be the first of a trilogy which describes not only the software but also the team and the hardware involved:

\begin{enumerate}
\item Wolfenstein 3D and the i386 CPU.
\item Doom and the i486 CPU.
\item Quake and the Pentium CPU.
\end{enumerate}

\bigskip

At first it may seem like a waste of time to read those ``old'' engine dedicated to extinct machines, compilers and operating systems. But there are tremendous values in them: Not only they are packed with clever tricks, they remind us of the constraints programmers from the past had to overcome. They remind us of the spirit it took to reach new frontiers. That spirit never died.\\
\par
I started to program in the years 2000s and was dismayed to have missed the golden era of the 90s where developers wrote their own game engine from scratch (idtech serie, build, src, goldSrc, unreal, dark, trespasser, ...). Only to catch on the iOS and Android revolution during which I enjoyed some success, even writing an app that was \#1 worldwide in 2009. I later found out that others felt the same about their time. John Carmack described the same feeling during his Fellowship, BAFTA speech:\\
\par
 \begin{fancyquotes}
I can remember when I was a teenager: I thought I had missed the golden age of 8 bits Apple II gaming. And I was never gonna be Richard Garriot. And time went by and I got to make
    my own marks and things after that.\\
    
    The opportunities that I had are not there for people today but they are new and better one.
 \bigskip
 \end{fancyquotes}

\par
 If anything, I hope this book will remind to people who struggle today that others have struggled before. You are not alone. Believe in yourself, in what you are passionate about and keep on looking for The Right Think to Do\footnote{"Hackers: Heroes of the Computer Revolution" by Steven Levy".}!\\



% \pagebreak
% \begin{fancyquotes}
% If you do what you love, and do it as well as you can, good things will eventually come of it. Not necessarily quickly or easily, but if you stick with it, they will come.
% \par
% - Michael Abrash: Graphic Programming Black Book\bigskip
%  \end{fancyquotes}
% \pagebreak  
\end{document}
