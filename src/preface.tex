\documentclass[book.tex]{subfiles}
\begin{document}


For the past 10 years, I have been writing articles explaining the internals of game engines. It all started back in 1999 when I downloaded the freshly opened source code of \mbox{Quake World} and eagerly opened it with \codeword{Visual Studio .NET}. After a few days of struggle I deleted the \codeword{quakesrc} folder, unable to make sense of anything and discouraged.

\bigskip

A few years later I came across legendary programmer and technical writer Michael Abrash's Graphics Programming Black Book. His beautiful articles described the essential algorithms and data structures used in Quake engine. Armed with this precious knowledge I went back to the code and read it all to the deep down.

\bigskip

This initial success was the sparkle that allowed me to associate hard work with deep knowledge. I thought many may be like me: Just in need of a roadmap to get them going. So I started to write ``source  code reviews'' and published them on 
\href{http://fabiensanglard.net}{\textit{fabiensanglard.net}}. Over the years, I proceeded to write more than fifty articles,  picking legendary games such as Doom, Quake or Out Of This World. I would open the source code, explore the subsystems, understand the overall architecture and draw a map that hopefully saves a lot of time to other adventurous programmers.

\bigskip

Writing and sharing my knowledge was a rewarding experience. Not only the feedback from readers was overwhelmingly positive, it also allowed me to improve a very precious skill: How to explain complicated things in simple terms. I have learned that good explanations only need three things: 
\begin{itemize} 
\item A very deep understanding of the subject. 
\item As many and as pretty drawings you can make\footnote{"Code: The Hidden Language of Computer Hardware and Software" by Charles Petzold is a superb example.}.
\item Dedication to sharing knowledge.
\end{itemize} 

\bigskip

Eventually I decided to take articles to the next level and came up with this book which I hope will be the first of a series about revolutionary game engines. 

\bigskip

At first, it may seems like a waste of time to read ``old'' source codes but I have found  tremendous values in them. Not only they teach clean design, provide a deeper understanding of moder hardware and software, trigger new ideas and inspire innovation, they also remind us of the talent and determination it once took to make a dream come true. They embody what creativity and dedication can produce.

\bigskip

I hope this book will help you to understand the tremendous effort it was to create Wolfenstein 3D. I hope it will inspire you to emulate yourself and keep on looking for the Right Thing to Do\footnote{"Hackers: Heroes of the Computer Revolution" by Steven Levy.}.
\end{document}