\documentclass[book.tex]{subfiles}
\begin{document}

For the past ten years, I have been writing articles explaining the internals of game engines. It all started back in 1999 when I downloaded the freshly open sourced code of \mbox{Quake} and eagerly opened it with \codeword{Visual Studio 6.0}. After a few days of struggling, I deleted \codeword{quakesrc} folder, discouraged and unable to make sense of anything.

\bigskip

A few years later I came across the legendary Graphics Programming Black Book by Michael Abrash. His articles explained the big picture of the Quake engine and detailed the now famous Binary Space Partition system\footnote{Chapter 59 in Michael Abrash Graphic Programming Black Book.}, Potentially Visible Set\footnote{Chapter 64 in Michael Abrash Graphic Programming Black Book.} and its compression techniques. Now knowing what to expect, I went back to the code and understood it deep down.

\bigskip

I thought many other programmers may be like me: capable but discouraged by apparent complexity. So I started to write ``source  code reviews'' and uploaded them on my
\href{http://fabiensanglard.net}{\textit{website}}. Over the years, I wrote more than fifty articles, selecting legendary games such as Doom, Quake, or Out Of This World. I would open the engine, explore the subsystems and the overall architecture, and draw a map that hopefully sparked interest and encouraged other adventurous programmers.

\bigskip

Sharing my knowledge was a rewarding experience. Not only was the feedback from readers positive, but explaining something in simple terms is an excellent way to make sure one masters a topic. It also tremendously improved both my capacity to ingest a large volume of code and my communication skills. I learned to rely extensively on drawings (a picture is worth $2^{20}$ words\footnote{"Code: The Hidden Language of Computer Hardware and Software" by Charles Petzold is a superb example.}) and as a result this book features hundreds of them. These skills proved invaluable in a career which ultimately led me to working at Google on Android.

\newpage

Eventually I decided to take my articles to the next level and came up with this book, which covers the first of the three milestone hardware and engine combinations of the 90s:



\begin{enumerate}
\item Wolfenstein 3D (1992) and the i386.
\item Doom (1993) and the i486-DX2.
\item Quake (1996) and the Pentium.
\end{enumerate}

\bigskip

It may appear like a waste of time to read and write about ``old'' engines dedicated to extinct machines, compilers, and operating systems, but they carry tremendous value. Not only are they packed with clever tricks, they also remind us of the constraints programmers from the past had to overcome. They remind us of the spirit it once took to reach new frontiers.\\
\par
Things have not changed much. These days we may deal with gigabytes, dedicated hardware accelerators, and multi-cores CPU but the spirit it takes to keep on moving forward remains the same. To those who struggle today, keep in mind you are not alone. Others have struggled before. Some have found fame and some have found fortune but in the grand scheme of things we all belong to a family of people who roll up their sleeves and try to make things better with hard work. Wherever it takes you, be proud of your labor. Be proud of your passion and keep on looking for The Right Thing to Do\footnote{Hackers: Heroes of the Computer Revolution" by Steven Levy.}!\\

\end{document}
