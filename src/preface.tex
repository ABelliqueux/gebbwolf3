\documentclass[book.tex]{subfiles}
\begin{document}

For the past 10 years, I have been writing articles explaining the internals of game engines. It all started back in 1999 when I downloaded the freshly open sourced code of \mbox{Quake} and eagerly opened it with \codeword{Visual Studio .NET}. After a few days I deleted \codeword{quakesrc} folder, unable to make sense of anything and discouraged.

\bigskip

A few years later I came across legendary programmer and technical writer Michael Abrash's Graphics Programming Black Book. His marvellous articles gave away the big picture of Quake engine and detailed some of its algorithms and data structures. Equiped with this precious knowledge I went back to the code, read and understood it to the deep down.

\bigskip

I thought many other programmers may be like me: Capable but discouraged by apparent complexity. So I started to write ``source  code reviews'' and published them on 
\href{http://fabiensanglard.net}{\textit{fabiensanglard.net}}. Over the years, I proceeded to publish more than fifty articles, selecting legendary games such as Doom, Quake or Out Of This World. I would open the engine, explore the subsystems, understand the overall architecture and draw a map that hopefully sparkled interest and encouraged other adventurous programmers.

\bigskip

Sharing my knowledge was a rewarding experience: Not only the feedback from readers was overwhelmingly positive, to try to explain something in simple terms was also a way to make sure I mastered a topic. It also allowed me to become a better teacher and learn to rely extensively on good drawings. A picture
is worth more than a thousand words\footnote{"Code: The Hidden Language of Computer Hardware and Software" by Charles Petzold is a superb example.}. Those skills proved invaluable in my career which lead me to Google.

\bigskip

Eventually I decided to take my articles to the next level and came up with this book which I hope will be the first of a trilogy:

\begin{enumerate}
\item Wolfenstein 3D and the i386 CPU.
\item Doom and the i486 CPU.
\item Quake and the Pentium CPU.
\end{enumerate}

Focusing first on what the hardware was capable of and then how the software brought it to live.

\bigskip

It may seem like a waste of time to read those ``old'' engine dedicated to extinct machines, compilers and operating systems. But there are tremendous values in them: Not only they are packed with clever tricks, they remind us of the constraints programmers from the past had to overcome. They remind us of the spirit it took to reach new frontiers. That spirit never died, it is in all off us who keep on trying to build things. If anything, I hope this book will remind to people who struggle today that others have struggled before. You are not alone. Believe in yourself and keep on aiming for The Right Think to Do\footnote{The Right Thing To Do was first brought to fame and extensively discussed in "Hackers: Heroes of the Computer Revolution" by Steven Levy". It was also often mentioned in John Carmack journal using the finger protocol.} :) !\\

\bigskip
THIS DOS NOT BELONG HERE: delete everything after this!!\\
\bigskip

I started to program in the years 2000 and was dismayed to have missed the golden era of the 90s where developers wrote their own engine from scratch (idtech serie, build, src, goldSrc, unreal, dark, trespasser, ...). Only to catch on the mobile revolution where I did get to enjoy some success, even writing an app that was \#1 worldwide in 2009. I later found out that other developers felt the same about their era. John Carmack recently described the same feeling during his Fellowship, BAFTA speech:\\

 \begin{fancyquotes}
I can remember when I was a teenager: I thought I had missed the golden age of 8 bits Apple II gaming. And I was never gonna be Richard Garriot. And time went by and I got to make
    my own marks and things after that.\\
    
    The opportunities that I had are not there for people today but they are new and better one.
 \bigskip
 \end{fancyquotes}


    
\bigskip
    I agree: There is no golden era. There are only dreams and the difficult path of excellence leading to them. This source code is a testimony of the talent and determination it once took to make a dream become true. It inspired me to emulate myself and I hope it will benefit you as well.

\bigskip


  
\end{document}
