\documentclass[book.tex]{subfiles}
\begin{document}

For the past ten years, I have been writing articles explaining the internals of game engines. It all started back in 1999 when I downloaded the freshly open sourced code of \mbox{Quake} and eagerly opened it with \codeword{Visual Studio 6.0}. After a few days of struggle, discouraged, I deleted \codeword{quakesrc} folder, unable to make sense of anything.

\bigskip

A few years later I came across the legendary Graphics Programming Black Book by Michael Abrash. His articles explained the big picture of Quake engine and detailed BSP\footnote{Binary Space Partition, Chapter 59 in Michael Abrash Graphic Programming Black Book.}, PVS\footnote{Probably Visiblet Set, Chapter 64 in Michael Abrash Graphic Programming Black Book.} and its compression techniques. Knowing what to expect, I went back to the code and understood it to the deep down.

\bigskip

I thought many other programmers may be like me: Capable but discouraged by apparent complexity. So I started to write ``source  code reviews'' and uploaded them on my
\href{http://fabiensanglard.net}{\textit{website}}. Over the years, I wrote more than fifty articles, selecting legendary games such as Doom, Quake or Out Of This World. I would open the engine, explore the subsystems, understand the overall architecture and draw a map that hopefully sparkled interest and encouraged other adventurous programmers.

\bigskip

Sharing my knowledge was a rewarding experience: Not only the feedback from readers was overwhelmingly positive, to explain something in simple terms is an excellent way to make sure one masters a topic. It also helped me to improve my communication, capacity to ingest large volume of code and learn to rely extensively on good drawings: A picture
is worth $2^{20}$ words\footnote{"Code: The Hidden Language of Computer Hardware and Software" by Charles Petzold is a superb example.}. Those skills proved invaluable in a career which lead me to Google and Android.

\bigskip

Eventually I decided to take my articles to the next level and came up with this book which I hope will be the first of a trilogy dedicated to three milestone computers and engines:

\begin{enumerate}
\item Wolfenstein 3D and the Intel 386 CPU.
\item Doom and the Intel 486 CPU.
\item Quake and the Intel Pentium CPU.
\end{enumerate}

\bigskip

At first it may seem like a waste of time to read those ``old'' engines dedicated to extinct machines, compilers and operating systems. But there are tremendous values in them: Not only they are packed with clever tricks, they remind us of the constraints programmers from the past had to overcome. They remind us of the spirit it took to reach new frontiers.\\ 
\par
If anything, I hope this book will remind to all of us who struggle today that others have struggled before. You are not alone. Some have found fame, some have found fortune. But we all belong to a community of people who roll up their sleeves and try to make things better with hard work. Wherever your work take you, be proud of it. Be proud of your passion and keep on looking for The Right Thing to Do\footnote{"Hackers: Heroes of the Computer Revolution" by Steven Levy".}!\\



% \pagebreak
% \begin{fancyquotes}
% If you do what you love, and do it as well as you can, good things will eventually come of it. Not necessarily quickly or easily, but if you stick with it, they will come.
% \par
% - Michael Abrash: Graphic Programming Black Book\bigskip
%  \end{fancyquotes}
% \pagebreak  
\end{document}
