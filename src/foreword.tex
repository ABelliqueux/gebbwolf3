Fabien's commentary on the classic game engine codebases have been a wonderful resource on the web, so I was thrilled that he decided to start expanding them all the way to book length. While often overshadowed by Doom, Wolfenstein 3D does hold a significant place in video game history, and it remains fun to run around in today, just like dropping a quarter in a Pac Man machine.\\
\par
 Despite being open source, the 16 bit code and assembly language is not easy to build or experiment with, so far fewer people have looked into it than the later codebases. The most remarkable thing about the project from today's perspective is just how small it was: one little directory of code files with no external dependencies.  Back then I barely trusted (with some reason!) the C standard library implementations that we had to work with, so almost everything was done in those few source files.\\
\par
I was 21 years old when I wrote most of the code, and I had only been programming in C for a year, so it is far from a masterpiece of coding style, but there are still some things that were done well. Compiled scalers are a case of code specialization taken to the extreme, which, combined with the low level VGA trick of multi column writes and the progressive performance characteristics of ray casting make it much more even in framerate than a conventional approach. The choice to use ray casting was also an important pragmatic decision.  I wasn't experienced enough yet to do a solid implementation of a polygon, or even line based, engine.  Ray casting got me where I needed to be at an acceptable cost.\\
\par 
So, set the way-back machine for 1992.\\
\par
ACHTUNG!\\
\par
-- John Carmack
