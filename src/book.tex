\documentclass[8pt]{book}

% All styles and package and commands are defined here.
% This allow to have one preambule used in all |/subfile| .
\usepackage{mystyle}

\begin{document}


\def\GE{} % Title
\def\BB{} % Title
\title{\GE \BB}

\def\Me{Fabien Sanglard} % Title
\author\Me

\definecolor{title_white}{rgb}{1,1,1}
\definecolor{title_grey}{rgb}{0.6,0.6,0.6}
       
\def\Version{0.1}

    %Cover page
    \subfile{cover}
    
    \renewcommand{\rmdefault}{cmss} 
    \renewcommand{\familydefault}{\sfdefault}
    
    % Table of content
    \addtolength{\cftsecnumwidth}{10pt}
    \tableofcontents
    
    \pagebreak
    \chapter*{Preface}
      \subfile{preface}

    \addcontentsline{toc}{subsection}{Preface}
    \chapter{Introduction}
      \subfile{intro}
    \chapter{Hardware}
      \subfile{hardware}
    \chapter{Team}
      \subfile{team}
      
    \chapter{Software}
       \subfile{software_architecture}
       \subfile{software}
       \subfile{software_2d}
       \subfile{software_3d}
       \subfile{software_vsd}
       \subfile{dying}
       \subfile{performances}
    \chapter{After Woftenstein 3D}
        \subfile{fallout}
        \subfile{wolf3dvr}
        \subfile{rott}

    \appendix
    \appendixpage

    \chapter{Let's compile like it's 1992}
    \chapter{The 640KB Barrier}\label{chap:barrier640}
        \subfile{mem_barrier}
    \chapter{CONFIG.SYS and AUTOEXEC.BAT}
         \subfile{config_autoexec}
    \chapter{Good stuff}
          \subfile{good_stuff}
    \chapter{Chocolate Wolfenstein 3D}
    \chapter{Release Notes by John Carmack}
        \subfile{release}
    \chapter{20 years anniversary commentary by John Carmack}
        \subfile{20years}
    
    \chapter{File formats}
    
        \subfile{maps}
        \subfile{3D_assets}    
    
        \subfile{2D_assets}            
        \subfile{music}
        \subfile{sound_effects}                    
    
    \chapter{TODO}
    Show VGA layout (screen and 4 planes)\\
    show asset compression\\
    perfs (fps)\\
demo recording not working ?\\
How much conventional memory is used?\\
show debug mode\\
Castle Wolfenstein\\
In sound part, trivia with a bunch of translations: https://www.gamefaqs.com/pc/564603-wolfenstein-3d/faqs/1824\\
Talk about morse code in episode 3 and 6, echo with romero head in doom2\\
Talk about how long it took them to make a level.\\
talk about cos and sin tables\\
virtualreality\\
in architecture, show files available next to engine.\\
a blast from the past with monkey island music evolution\\
introduce fixed point with raytracing\\
show double loop ping pong pseudo code used for raytracing\\
talk about 3DO port: http://www.vgmpf.com/Wiki/index.php?title=Wolfenstein\_3D\_(3DO)
introduce latches in menu screen. Explain they were intended to other graphic mode 16 colors (1 bit per plan, 4bits per pixel)\\
Ask john why he did not use 320*240 which offered square pixels.\\

hint manual was done on NeXT. Nothing else was done with it (they got their NeXT in december 1991.\\
\par
After talking with Romero and Tom, Scott learned that it was taking the group only about one day to make a level of the game. Ka-chung! Dollar signs! Instead of just three episodes, why not have six? Scott said, “If you can do thirty more levels, it would only take you fifteen days. And we could have it where people could buy the first trilogy for thirty-five dollars or get all six for fifty dollars, or if people buy the first episode and later want the second episodes it will be twenty dollars. So there’s a reason to get them all!” After some consideration, id agreed.\\
\par
show previous FPS they did (hovertank and catacombs 3D\\

Tom had to really push to get pushwalls (hidden walls).
\subfile{cover_back}
\end{document}