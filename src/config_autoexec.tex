\documentclass[book.tex]{subfiles}
\begin{document}
Upon startup, the operating system read two files automatically from the booting device (which could be either the hard-drive(C:) or floppy disk (A:)
CONFIG.SYS:\\
 \begin{breakable_box}
   \begin{verbatim}
DEVICE=C:\WINDOWS\HIMEM.SYS
DOS=HIGH,UMB
DEVICEHIGH=C:\WINDOWS\EMM386.EXE AUTO RAM
DEVICE=C:\WINDOWS\MOUSE.SYS
\end{verbatim}
\end{document}
\\
AUTOEXEC.BAT:\\
 \begin{breakable_box}
   \begin{verbatim}
@echo off
SET SOUND=C:\PROGRA~1\CREATIVE\CTSND
SET BLASTER=A220 I5 D1 H5 P330 E620 T6
SET PATH=C:\Windows;C:\ 
LH C:\Windows\COMMAND\MSCDEX.EXE /D:123
\end{verbatim}
\end{document}

Trivia: To this day, the old way to refer to drives is still in Windows (C:):
\begin{itemize}
\item A: used to be the floppy drive containing the program code.
\item B: the floppy drive containing the data.
\item C: became used when hard drives became more wide-spread.
\item D: was adopter when CD-ROM reader apperared later.
\end{itemize}
\end{document}