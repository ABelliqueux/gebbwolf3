\documentclass[book.tex]{subfiles}
\begin{document}
The problem with conventionnal memory limitation was so bad that most games has to ship with explanations about how everything worked: Here is an extract from W3DHELP.EXE.\\

 \begin{breakable_box}
   \begin{verbatim}
THE 640K BARRIER
================

This section isn't actually needed in order to get our programs running.
What is contained in here is for the most part background information to
better assist our customers in understanding why they need to make more
conventional memory available.

When MicroSoft first made DOS 1.0, 640 kilobytes (KB) was set aside as
the highest amount of memory that a computer could have. The 640KB of
memory is what is called "conventional memory". To maintain compatibi-
lity with older versions, this was never changed. Advances in memory 
management have made access to memory beyond 640KB, but this memory can
only hold data; the program actually has to run in the first 640KB. 
This first 640k is called "Conventional Memory".

Here is a brief discussion of the different types of memory available
on your computer. The most important one is Conventional memory.

CONVENTIONNAL MEMORY starts at 0k and normally ends at 640k. (The 
instances where this is not the case are EXTREMELY rare) If you are not
using some sort of memory manager (such as DOS's EMM386, Quarterdeck's
 QEMM or Qualitas'386MAX), this is the only type of memory you have.
Conventional memory is used by DOS as well as device drivers and TSR's
(Terminate and Stay Resident Programs). A TSR is a program that is loaded
into your computer's memory (usually from the CONFIG.SYS or AUTOEXEC.BAT
files) and stays there. Host programs remove themselves from memory after
execution, a TSR does not. Device drivers and TSR's are programs that
enable the computer to use additional hardware such as a mouse, scanner,
CD-ROM, expanded or extended memory, etc. A program such as an Apogee 
game is NOT a program that can be loaded as a TSR. If all you have is
conventional memory, anything that you would load as a TSR would come out
of this section of memory. Take too much away, and you're not left over
withenough memory to run our product. 

If you are getting an out of memory error from our program, it is this
memory that you are running out of. Whether you have 1 meg, 8 meg of
memory, or 32 meg of memory, it's irrelevant. Only the first 640k of
memory is available for program execution. Please do not confuse this
with hard drive space. Your hard drive space is not memory, and is not
relevant nor should be considered in this example.

UPPER MEMORY starts at 640k and ends at 1024k. Normally, this area is
used for things such as system ROM, video and hardware cards, and the
like. On most PC's hardware does not use the entire upper memory area,
and with the use of the aforementioned memory managers, (EMM386, QEMM,
386MAX, etc.) you can move some TSR's into this memory area. These 
unused areas are called Upper Memory Blocks (UME'S), and this is where
some TSR's can be loaded.

EXTENDED MEMORY (XMS) is the memory addressed above 1024k. Extended 
memory requires the use of a memory manager, such as MS/DOS's HIMEM.SYS.
This region of memory is not usable for standard program execution; it
can only be used for data storage. Aogee programs that use this type of
memory(such as Wolfenstein & Blake Stone), only use this to store level
or graphic data. The actual program itself is running in conventional
memory. 

HIGH MEMORY HREH (HMH) is the first 64k of extended memory. This is a
special region of memory that is most commonly used to load DOS high. 
When you issue the DOS:HIGH command in your config.sys file, the amount 
of conventional memory that was previously being occupied by DOS itself 
is moved into this region.

EXPANDED MEMORY (EMS) is another type of memory that some MS/DOS programs
can make use of. Like XMS, this memory is not available for program
execution, it's only used for data storage due to it's nature. An
explanation of this type of memory is rather technical, so it will not be
delved into here. If you're curious, check your DOS manual, or your memory
manager manual.


When you first start up your computer, there are two files that your
computer looks at: CONFIG.SYS and AUTUEXEC.BAT. These two files contain
lists of device drivers and TSR's that are automatically run when starting
your computer. Each of these takes up space, and it is taken away from the
640k of conventional memory. As more and more programs are loaded from the
autoexec.bat and config.sys files, you have less and less available from
the original 640k. Since it is this memory that programs run in, you can
see that the amount taken away from the programs executed in config.sys
and autoexec.bat would want to be kept to a minimum. This can be
accomplished by either reducing the amount of programs loaded in from 
config.sys and autoexec.bat, or moving them to high memory via the use of
EMM386, QEMM, 386MAX, or some other memory management program.   
   \end{verbatim}

 \end{breakable_box}
\end{document}
