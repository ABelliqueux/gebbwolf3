\documentclass[book.tex]{subfiles}
\begin{document}
Wolfenstein 3D ended up a colossal financial success. Despite having multiple episodes, each made of 10 levels:\\
\par
\begin{itemize}
\item Episode 1: Escape from Wolfenstein
\item Episode 2: Operation: Eisenfaust
\item Episode 3: Die, Fuhrer, Die!
\item Episode 4: A Dark Secret
\item Episode 5: Trail of the Madman
\item Episode 6: Confrontation
\end{itemize}
The game was not long enough to satisfy many of its fans. id Software contracted other companies to create more maps and also worked on a sequel of their own.

\section{Spear of Destiny}
Released September 18, 1992 and entirely done by id Software, Spear Of Destiny used the same game engine but with new graphics, music, and levels. It was a prequel to the adventures of Wolfenstein 3D's hero B.J. Blazkowicz and consisted of one additional episode made of 21 levels.\\
   \par
\begin{figure}[H]
\centering
 \fullimage{spears_of_destiny_intro.png}
 \end{figure}
 \par
 To work on a sequel made a lot of sense:
 \begin{enumerate}
 \item Sales were so good it would have been silly not to take advantage of the momentum.
 \item It gave the engineering team time to build the next generation of engine and tools while keeping the design and graphics teams busy.
 \end{enumerate}
Despite using the same engine, Spear Of Destiny was innovative in introducing huge transparent sprites to simulate vegetation and also attempted to break away from the orthogonal world constraint with clever map design leveraging all of each map's real estate.
    \par
\begin{figure}[H]
\centering
 \fullimage{spears_of_destiny_play.png}
 \end{figure}
 \par


   \par
\begin{figure}[H]
\centering
 \fullimage{spears_of_destiny_creative_map.png}
 \end{figure}
 \par

To fight piracy, SOD shipped with a copy protection typical of the early 90s. Since copying a disk was easy but photocopies were much more cumbersome, the game would refuse to start unless the player managed to answer a question about something in the game manual.\\
    \par
\begin{figure}[H]
\centering
 \fullimage{spears_of_destiny_copy_protection.png}
 \end{figure}
 \par
 The copy protection mechanism featured backdoors which are probably private jokes.
\par
\begin{minipage}{\textwidth}
\lstinputlisting[language=C]{code/backdoor_words.c}
\end{minipage}
\par
And the associated answers:\\
\par
\begin{minipage}{\textwidth}
\lstinputlisting[language=C]{code/backdoor_answers.c}
\end{minipage}
\par
Notice the reference to the 1984 movie, Wargames.
    \par
\begin{figure}[H]
\centering
 \fullimage{spears_of_destiny_easter_egg.png}
 \end{figure}
 \par

\end{document}
