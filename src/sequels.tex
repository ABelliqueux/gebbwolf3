\documentclass[book.tex]{subfiles}
\begin{document}
Wolfenstein 3D was a colossal financial success. Despite having multiple episodes, each made of 10 levels, the game was not long enough to satisfy many of its fans:\\
\par
\begin{itemize}
\item Episode 1: Escape from Wolfenstein
\item Episode 2: Operation: Eisenfaust
\item Episode 3: Die, Fuhrer, Die!
\item Episode 4: A Dark Secret
\item Episode 5: Trail of the Madman
\item Episode 6: Confrontation
\end{itemize}
id Software contracted other companies to create more maps and also worked on a sequel of its own.

\section{Spear of Destiny}
Released on September 18, 1992 and entirely done by id Software, Spear Of Destiny used the same game engine but with new graphics, music, and levels. It was a prequel to the adventures of Wolfenstein 3D's hero B.J. Blazkowicz and consisted of one additional episode made of 21 levels.\\
   \par
\begin{figure}[H]
\centering
 \fullimage{spears_of_destiny_intro.png}
 \caption{Spear of Destiny splash screen.}
 \end{figure}
 \par
 Working on a sequel made a lot of sense:
 \begin{enumerate}
 \item Sales were good, so it would have been silly not to take advantage of the momentum.
 \item It gave the engineering team time to build the next generation of engine and tools while keeping the design and graphics teams busy.
 \end{enumerate}
Although it used the same engine, Spear Of Destiny was innovative in several ways. It introduced huge transparent sprites to simulate vegetation and attempted to break away from the orthogonal world constraint with clever map design leveraging all of each map's real estate.\\
    \par
On the opposite page, bottom figure, the red Angel Of Death boss showcases how the game thematic dramatically shifted away from the WW2 era. The "hero fighting demons with big guns" theme would be the corner stone of id Software next title, DOOM. 
\begin{figure}[H]
\centering
 \fullimage{spears_of_destiny_play.png}
 \end{figure}
 \par
 \begin{figure}[H]
\centering
  \scaledimage{0.85}{angel_of_death.png}
 \end{figure}



   \par
\begin{figure}[H]
\centering
 \fullimage{spears_of_destiny_creative_map.png}
 \caption{Spear of Destiny map attempted to look less rectangular.}
 \end{figure}
 \par

Wolfenstein 3D shipped with no copy protection mechanism. Upon receiving the registered version is was dead simple to copy the floppies and deprive id Software from a much deserved profit. To solve this problem, Spear Of Destiny shipped with a copy protection typical of the early 90s. Since copying a disk was easy but photocopy machine were much more difficult to come around, the game would refuse to start unless the player answered a question about something found in the game manual\footnote{Companies released increasingly elaborated paper based protection over the years. LucasArts box of "Monkey Island" contained "rolling wheels" which could generate 105 questions/answers. Delphine software relied on an elaborated color drawing for "Operation Stealth" (color photocopier were virtually non-existent). "Strike Commander" shipped with a 98 pages impossible to photocopy fake "Sudden Death" magazine.}.
    
\begin{figure}[H]
\centering
 \fullimage{spears_of_destiny_copy_protection.png}
 \caption{Spear of Destiny "90s classic" copy protection.}
 \end{figure}
 \par
 The copy protection mechanism featured backdoors which are probably private jokes.\\
\par
\begin{minipage}{\textwidth}
\lstinputlisting[language=C]{code/backdoor_words.c}
\end{minipage}
\par
And the associated responses from the protection system:


\begin{minipage}{\textwidth}
\lstinputlisting[language=C]{code/backdoor_answers.c}
\end{minipage}\\
\par
Note the reference to the 1984 movie, WarGames which in its original version invited the user to play an unspecified game.
    \par
\begin{figure}[H]
\centering
 \fullimage{spears_of_destiny_easter_egg.png}
 \end{figure}
 \par

\end{document}
