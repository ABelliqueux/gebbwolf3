\documentclass[book.tex]{subfiles}
\begin{document}
The game ended up a big hit. Sales were so good the six original episodes were not enough to satisfy players:\\
\par
\begin{itemize}
\item Episode 1: Escape from Wolfenstein\\
\item Episode 2: Operation: Eisenfaust\\
\item Episode 3: Die, Fuhrer, Die!\\
\item Episode 4: A Dark Secret\\
\item Episode 5: Trail of the Madman\\
\item Episode 6: Confrontation\\
\end{itemize}
Sequels and port to other machines made business sense. id Software ended up developing some and contracting others. With one notorious port (iOS) coming out almost 20 years later in 2009.

\section{Spears of Destiny}
Released on September 18, 1992 Spears Of Destiny used the same game engine but with new graphics, music and levels. It is a prequel to the adventures of wolf hero: B.J. Blazkowicz. It consist of one additional episode of 21 levels.\\
   \par
\begin{figure}[H]
\centering
 \fullimage{spears_of_destiny_intro.png}
 \end{figure}
 \par
 The game resued Wolfenstein 3D engine, allowing a part of the team to release a second title while an other part of the team worked on the next technology to power DOOM. The title is nonetheless innovative since it introduces huge transparent sprites to simulate vegetation and also tries to break away from the orthogonal world with clever map design:
    \par
\begin{figure}[H]
\centering
 \fullimage{spears_of_destiny_play.png}
 \end{figure}
 \par


   \par
\begin{figure}[H]
\centering
 \fullimage{spears_of_destiny_creative_map.png}
 \end{figure}
 \par


 SOD features a copy protection mechanism: A set of random question which could only be answered with the game manual:\\
    \par
\begin{figure}[H]
\centering
 \fullimage{spears_of_destiny_copy_protection.png}
 \end{figure}
 \par
 An easter egg is hidden here. A serie of word are always a valid answer. Amond them "Joshua", a reference to 1983 movie WarGames:\\
    \par
\begin{figure}[H]
\centering
 \fullimage{spears_of_destiny_easter_egg.png}
 \end{figure}
 \par



Trivia: Angel of Death looked like a demon and was a good insight into what would come next: Doom.




It was essentially a mod of Wolfenstein 3D. SOD helps to put things into perspective when it comes to being open source and allow people to figure out the file formats. SOD may not have sold as well had the game been open.\\
No a shareware version  but  2-level playable demo was distributed\\
Lost Episodes, Each of these, consists of 21 levels. New level textures, new enemies, and new appearances for old enemies. Published by FormGen Corporation in May 1994.''
\\
\section{Mission Pack}
Mission Packs(released by FormGen Corporation in May 1994, resemble many fan-made mods a.k.a: Lost Episodes):\\
Mission 2: Return to Danger\\
Mission 3: Ultimate Challenge\\




\end{document}
