\documentclass[book.tex]{subfiles}
\begin{document}
Wolfenstein 3D ended up a collosal financial success. Despite its many episodes, each made of 10 levels:\\
\par
\begin{itemize}
\item Episode 1: Escape from Wolfenstein
\item Episode 2: Operation: Eisenfaust
\item Episode 3: Die, Fuhrer, Die!
\item Episode 4: A Dark Secret
\item Episode 5: Trail of the Madman
\item Episode 6: Confrontation
\end{itemize}
The game was not long enough to satisfy the many players. id Software contracted other companies to make more maps and they also worked on a sequel of their own.

\section{Spears of Destiny}
Released on September 18, 1992 and entirely done by id Software, Spears Of Destiny used the same game engine but with new graphics, music and levels. It is a prequel to the adventures of wolf hero: B.J. Blazkowicz and consist of one additional episode made of 21 levels.\\
   \par
\begin{figure}[H]
\centering
 \fullimage{spears_of_destiny_intro.png}
 \end{figure}
 \par
 To work on a sequel made a lot of sense:
 \begin{enumerate}
 \item Sales were so good it would have been silly not to take advantage of the momentum and provide the market with something in strong demand.
 \item It gave time to the engineering team to build the next generation of engine and tools while keeping the design and graphic team busy.
 \end{enumerate}
Despite using the same engine, Spear Of Destiny is still innovative since it introduces huge transparent sprites to simulate vegetation and also tries to break away from the orthogonal world contraint with clever map design leveraging all the map real estate.
    \par
\begin{figure}[H]
\centering
 \fullimage{spears_of_destiny_play.png}
 \end{figure}
 \par


   \par
\begin{figure}[H]
\centering
 \fullimage{spears_of_destiny_creative_map.png}
 \end{figure}
 \par

To fight piracy, SOD shipped with a copy protection typical of the early 90s. Since copying a disk was easy but photocopies were much harder to do, the game would refuse to start unless the player managed to answer a question about something in the game manual.\\
    \par
\begin{figure}[H]
\centering
 \fullimage{spears_of_destiny_copy_protection.png}
 \end{figure}
 \par
 The copy protection mechanism features backdoors which are probably private jokes.
\note{What does "spoon" means, what about eveyrthing else ?!}
\par
\begin{minipage}{\textwidth}
\lstinputlisting[language=C]{code/backdoor_words.c}
\end{minipage}
\par
And the associated answers:\\
\par
\begin{minipage}{\textwidth}
\lstinputlisting[language=C]{code/backdoor_answers.c}
\end{minipage}
\par
Notice the reference to the 1984 movie, Wargames.
    \par
\begin{figure}[H]
\centering
 \fullimage{spears_of_destiny_easter_egg.png}
 \end{figure}
 \par

\end{document}
